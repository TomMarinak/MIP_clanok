\documentclass[10pt,a4paper]{article}

\usepackage[slovak]{babel}
\usepackage[utf8]{inputenc}
\usepackage{graphicx}
\usepackage{amsmath}
\usepackage{hyperref}
\usepackage{cite}
\usepackage{acronym}

\usepackage[a4paper, top=2.5cm, bottom=2.5cm, left=2.5cm, right=2.5cm]{geometry}

\bibliographystyle{unsrt}

\title{Mass Use of Drones for Civilian Purposes. (\textbf{MDCU})\thanks{Semestrálny projekt v predmete Metódy inžinierskej práce, ak. rok 2024/25, vedenie: Vladimír Mlynarovič}}
\author{Tomáš Mariňák, Dmytro Seredenko, Dominik Luksaj\\[2pt]
	{\small Slovenská technická univerzita v Bratislave}\\
	{\small Fakulta informatiky a informačných technológií}\\
	{\small \texttt{xseredenkod@stuba.sk xmarinak@stuba.sk xluksaj@stuba.sk}}}
\date{\small \today}

\begin{document}
\maketitle

\section{Anotacia}

 Drony sa dnes používajú na rôzne účely, ako napríklad doručovanie tovaru, poskytovanie prvej pomoci, monitorovanie bezpečnosti a dokonca aj pri boji s terorizmom. Ich výhoda spočíva v rýchlosti pohybu, schopnosti zbierať presné údaje a doručovať služby tam, kde je to ťažké zabezpečiť tradičnými metódami. To otvára nové možnosti v každodennom živote a môže zlepšiť efektivitu mnohých procesov.

 V oblasti dopravy môžu drony rýchlo doručovať zásielky, čím sa šetrí čas a náklady. V medicíne môžu priniesť lieky alebo rýchlo reagovať na núdzové situácie. Drony sú tiež užitočné pri monitorovaní a pátracích operáciách, či už na ochranu verejnosti alebo pri riešení hrozieb.

 Zámerom nášho projektu je vytvoriť decentralizovaný systém, ktorý by umožnil využívať drony na tieto účely v širokom rozsahu. Predstavujeme si sieť dronov rozmiestnených po meste, ktoré budú zabezpečovať ich správu a fungovanie. Tento systém bude flexibilný a efektívny a umožní dronom poskytovať služby v rôznych oblastiach, ako je doprava, bezpečnosť a pomoc v núdzi.

\end{document}

