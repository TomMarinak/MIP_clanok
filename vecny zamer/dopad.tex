\documentclass[10pt,a4paper]{article}

\usepackage[slovak]{babel}
\usepackage[utf8]{inputenc}
\usepackage{graphicx}
\usepackage{amsmath}
\usepackage{hyperref}
\usepackage{cite}
\usepackage{acronym}
\usepackage[a4paper, top=2.5cm, bottom=2.5cm, left=2.5cm, right=2.5cm]{geometry}
\usepackage{array}  
\usepackage{booktabs}

\setlength{\parskip}{15pt}
\pagestyle{headings}

\title{Masové používanie dronov na civilne účely. (\textbf{MDCU})\thanks{Semestrálny projekt v predmete Metódy inžinierskej práce, ak. rok 2024/25, vedenie: Vladimír Mlynarovič}} % meno a priezvisko vyučujúceho na cvičeniach
\author{Tomáš Mariňák, Dmytro Seredenko,  Dominik Luksaj\\[2pt]
	{\small Slovenská technická univerzita v Bratislave}\\
	{\small Fakulta informatiky a informačných technológií}\\
	{\small \texttt{xseredenkod@stuba.sk xmarinak@stuba.sk xluksaj@stuba.sk}}
\date{\small \today}}


\begin{document}

\maketitle

\section{DOPAD}

V následujúcej časti sa venujeme hodnoteniu dopadov masového nasadenia dronov a bezpilotných vzdušných prostriedkov (ďalej len UAV) na rôzne oblasti spoločenského, technologického a ekonomického života, so zameraním na ich prínosy pre infraštruktúru, logistiku a rozvoj moderných služieb, no taktiež na problemy s legislatívou a technickymi/fyzickými limitami dronov/UAV.

\subsection{Prínosy projektu na rozvoj poznania, aplikovaného výskumu a inovácií}
Projekt \textbf{MDCU} zameriava na detailné preskúmanie možností integrácie veľkého množstva dronov do mestského a prímestského prostredia, pričom projekt prináša nové postupy riadenia, navigácie a dohľadu nad autonómnymi letmi. Drony sa v súčasnosti používajú najmä individuálne, no masové využitie si vyžaduje nové vedecké prístupy, ktoré integrujú prvky umelej inteligencie, senzorických systémov a moderných komunikačných technológií.

Významným prínosom projektu je vývoj algoritmov pre koordináciu veľkého počtu dronov vo vzdušnom priestore. Ide najmä o plánovanie letových trás, predikciu kolíznych situácií, adaptívnu reakciu na prekážky a optimalizáciu energií. Projekt tak prispieva k vytváraniu tzv. U-space, čo je európsky koncept pre správu nízkoletového vzdušného priestoru. Zároveň rozvíja aj úplne nové prístupy v oblasti distribuovanej umelej inteligencie, kde jednotlivé drony spolupracujú ako inteligentný systém.

Inovačný význam je viditeľný aj v oblasti senzoriky. Projekt skúma nové druhy optických, termálnych a LIDAR senzorov optimalizovaných pre autonómny let, čo umožní bezpečnejšie operácie aj v komplikovaných podmienkach – nočné prostredie, mestské ulice, okolie infraštruktúry či miesta zníženej viditeľnosti.

Dôležitým výstupom je aj tvorba nových metodík pre implementáciu dronových služieb v každodenných civilných procesoch: doručovanie zásielok, monitoring kritickej infraštruktúry, automatizovaná dokumentácia, prvá pomoc, hodnotenie škôd po živelných udalostiach či zabezpečenie verejnej bezpečnosti. Tieto postupy môžu slúžiť ako štandardizované odporúčania pre firmy, mestá a štátne inštitúcie.

Projekt tak významne rozširuje súčasný stav poznania technologického aj spoločenského charakteru a tvorí pevný základ pre ďalší aplikovaný výskum v oblasti inteligentnej mobility.


\subsection{Spôsob a miera využiteľnosti výsledkov pre žiadateľa v SR a v zahraničí}
Výsledky projektu sú široko využiteľné pre organizácie pôsobiace v rôznych oblastiach. V Slovenskej republike je to najmä logistický sektor, poskytovatelia kuriérskych služieb, zdravotnícke zariadenia, záchranné jednotky a správcovia energetickej a dopravnej infraštruktúry. Pre tieto subjekty projekt predstavuje novú generáciu technológií, ktoré umožnia automatizovať procesy, znížiť náklady a zvýšiť bezpečnosť.

V oblasti samospráv môžu byť výsledky využité pri zavádzaní inteligentných mestských služieb, ktoré zahŕňajú napríklad pravidelný prehľad dopravy, monitorovanie životného prostredia, sledovanie výstavby alebo hodnotenie rizikových oblastí. Drony s pokročilými senzormi môžu zásadne zjednodušiť správu mestských území a priniesť efektívnejšie rozhodovanie úradov.

V zahraničí sú výsledky projektu kompatibilné s aktuálnymi štandardmi Európskej agentúry pre bezpečnosť letectva (EASA). To umožní plnú integráciu vyvinutých riešení do európskeho U-space a ich rýchle nasadenie v krajinách Európskej únie. Projekt preto podporuje export technológií a know-how, čím môže prispieť k posilneniu postavenia slovenských výskumných inštitúcií a firiem na medzinárodnej úrovni.

Potenciál využitia je však globálny – riešenia možno aplikovať v oblastiach s vysokou urbanizáciou, pri zásahoch v nedostupnom teréne či pri koordinácii humanitárnej pomoci. Projekt tak môže prispieť aj k zahraničným výskumným spoluprácam a môže sa stať základom ďalších medzinárodných iniciatív.


\subsection{Miera ekonomického prínosu}
Ekonomické prínosy projektu sú významné a rozmanité. V prvom rade ide o podporu technologického sektora v SR, ktorý môže priamo profitovať z vývoja nových zariadení, softvéru a analytických nástrojov. Firmy pôsobiace v oblasti hardvéru môžu využiť nové konštrukčné a senzorické riešenia, zatiaľ čo softvérové spoločnosti môžu stavať na algoritmoch a metódach riadenia dronov, ktoré projekt prináša.

Masové nasadenie dronov prináša výrazné ekonomické úspory. V logistike môžu drony skrátiť čas doručovania a znížiť potrebu časti pozemnej dopravy, čo vedie k zníženiu nákladov na palivo, servis a ľudské zdroje. V monitorovacích činnostiach nahrádzajú drony tradičné postupy založené na manuálnom zbieraní dát, vysielaní vozidiel či využití vrtuľníkov, ktoré sú finančne aj energeticky náročné.

Významný je aj prínos pre zamestnanosť. Aj keď drony automatizujú niektoré činnosti, zároveň vytvárajú nové profesie — operátor autonómnych systémov, analytik letových dát, dizajnér senzorických modulov, špecialista U-space infraštruktúry, servisný technik či odborník na kyberbezpečnosť vzdušných systémov. Projekt tak nepriamo podporuje vznik nových pracovných miest s vyššou pridanou hodnotou.

V neposlednom rade má projekt aj pozitívny environmentálny dopad. Vďaka elektrickému pohonu a vysokej efektivite dopravy prispievajú drony k znižovaniu emisií skleníkových plynov. Ekonomický prínos tu spočíva nielen v úsporách, ale aj v možnosti implementovať ekologickejšie procesy, ktoré sú v súlade s trendmi udržateľného rozvoja.


\subsection{Opatrenia na maximalizáciu výsledkov, dopadu a komunikáciu výstupov}
Aby boli výsledky projektu maximálne využité, je potrebné dôsledné plánovanie ich prenosu do praxe. Projekt počíta s intenzívnou komunikáciou smerom k výskumným inštitúciám, odborným partnerom aj verejnosti.

Výstupy projektu budú prezentované prostredníctvom odborných workshopov, konferencií a praktických demonštrácií, ktoré umožnia potenciálnym záujemcom priamo vidieť fungovanie vyvinutých riešení. Súčasťou komunikácie bude aj tvorba technickej dokumentácie, metodík pre mestá a poskytovateľov služieb, a rovnako aj tvorba otvorených datasetov, ktoré môžu ďalej využívať výskumníci.

Projekt počíta aj s vybudovaním partnerstiev so slovenskými aj zahraničnými organizáciami. Tieto partnerstvá umožnia rýchle zavedenie výsledkov do praxe a zabezpečia ich škálovanie. Súčasťou je aj koordinácia s regulačnými orgánmi, aby sa nové technológie mohli integrovať v súlade s legislatívou.

Dôležitou súčasťou maximalizácie dopadu je aj práca s verejnosťou. Keďže drony sú technológia s výrazným spoločenským dopadom, projekt bude aktívne vysvetľovať ich výhody, bezpečnostné mechanizmy a prínos pre spoločnosť. Transparentná komunikácia podporí dôveru verejnosti v nové služby, čo je kľúčové pre ich budúce nasadenie vo veľkom rozsahu.


\subsection{Transportné systémy dronov – súčasný stav a limity}
Transportná oblasť je jednou z najperspektívnejších aplikácií civilných dronov, avšak v praxi je stále v štádiu intenzívneho vývoja. Viaceré technologické prototypy už umožňujú prepravu menších zásielok na vzdialenosť niekoľkých kilometrov, no pre ich masové nasadenie je potrebné vyriešiť viacero technických aj prevádzkových problémov.

Najväčším obmedzením je kapacita batérií, ktorá limituje dolet aj hmotnosť prenášaného nákladu. Súčasné komerčné drony dokážu preniesť spravidla 2–5 kg, pričom ich letový čas sa pohybuje okolo 20–30 minút. Pre logistické spoločnosti je to nedostatočné na plnohodnotnú náhradu časti cestnej dopravy, najmä v rozľahlých alebo členitých oblastiach.

Ďalším problémom je spoľahlivá navigácia v komplexnom mestskom prostredí. Drony musia dynamicky reagovať na pohyb ľudí, automobilov či neplánované prekážky a zároveň sa riadiť presnými letovými koridormi. Systémy vizuálnej lokalizácie či LIDAR senzory dokážu tieto problémy čiastočne riešiť, no stále dochádza k nepresnostiam, najmä pri sťažených svetelných podmienkach alebo vo veľmi úzkych priestoroch medzi budovami.

Nezanedbateľným problémom je aj integrácia veľkého množstva transportných dronov do existujúcej vzdušnej dopravy. Na efektívnu prevádzku je potrebná spolahlivá koordinácia s U-space službami, predikcia kolíznych trajektórií, komunikácia medzi dronmi a infraštruktúrou a jednotné bezpečnostné protokoly. Tieto služby sú stále vo vývoji a zatiaľ nie sú plošne implementované.

Hoci sa transportné drony rýchlo približujú praktickému nasadeniu, ich používanie v rozsiahlej logistickej sieti je stále limitované technologickými rizikami, nevyspelosťou navigačných algoritmov a nedostatočnou robustnosťou v extrémnych podmienkach. Projekt preto prispieva k vývoju riešení, ktoré majú tieto problémy dlhodobo eliminovať.


\subsection{Legislatívne problémy spojené s prevádzkou dronov a UAV}
Masové používanie dronov v civilnom priestore naráža aj na významné legislatívne bariéry. Európska legislatíva je harmonizovaná prostredníctvom EASA, avšak mnoho pravidiel je stále relatívne nové a ich implementácia v členských štátoch nie je jednotná.

Jedným z hlavných problémov je kategorizácia rizikovosti letov. Pre viaceré plánované aplikácie, najmä pre autonómny transport zásielok nad obývanými oblasťami, je nutné spĺňať podmienky kategórie specific. To si vyžaduje vypracovanie detailného SORA(Specific Operations Risk Assessment) hodnotenia, ktoré je pre menšie organizácie finančne aj administratívne náročné. Mnohé startupy či municipalitné projekty tak nemajú kapacity vykonať všetky potrebné právne kroky.

Ďalšou výzvou je povinnosť udržiavať dron v priamom vizuálnom kontakte (VLOS) pri väčšine civilných operácií. Pre autonómne transportné operácie je však nevyhnutná prevádzka mimo priamej viditeľnosti (BVLOS). Hoci legislatíva BVLOS umožňuje, vyžaduje to špeciálne povolenia, redundantné systémy bezpečnosti (automatické detekčno-vyhýbacie systémy), a často aj dohľad operátora, čo výrazne komplikuje komerčné nasadenie.

Problémom je tiež ochrana súkromia a osobných údajov. Pri masovom nasadení dronov môže byť nevyhnutné monitorovať veľké územia pomocou kamier a senzorov, čo môže byť v rozpore s GDPR a národnými zákonmi o ochrane osobných údajov. Na zamedzenie neželaného zberu dát si legislatíva vyžaduje technické obmedzenia kamerových systémov, zónovanie letov či anonymizáciu záznamov, avšak tieto opatrenia môžu priamo ovplyvniť kvalitu navigácie a spoľahlivosť služieb.

V neposlednom rade legislatíva rieši aj hluk, rušenie rádiových frekvencií a zodpovednosť za škodu. Pri masových transportných operáciách je potrebné vytvoriť jednoznačné pravidlá pre prípady technických zlyhaní, straty spojenia, kolízie alebo pádu dronu. Súčasné predpisy však nie sú ešte úplne prispôsobené pre situácie, keď by v jednej oblasti mohli operovať desiatky či stovky autonómnych UAV súčasne.

Pre riešenie týchto problémov je potrebné úzke prepojenie technologického vývoja, štandardizácie a legislatívnych úprav. Projekt sa preto zameriava aj na návrh odporúčaní pre regulačné orgány, ktoré môžu v budúcnosti výrazne zjednodušiť proces licencovania a zabezpečiť bezpečné, efektívne a zákonne kompatibilné používanie dronov.

Rastúci trend urbanizácie a zvýšené nároky na logistiku vytvárajú tlak na zavádzanie nových foriem mobility. Moderné smart cities sa preto zameriavajú na koncept urban air mobility (UAM), ktorý využíva bezpilotné prostriedky na rýchly presun ľudí, tovarov a dát. Podľa vybraných medzinárodných štúdií, medzi nimi aj „Integrating Urban Air Mobility into Smart Cities"\cite{UAM-DiVito}, sa očakáva, že drony v najbližších desaťročiach prevezmú kľúčovú úlohu v mestskej logistike, monitorovaní a dynamickom riadení mestských procesov.

Transportné systémy dronov sa už začínajú objavovať v pilotných projektoch, kde slúžia napríklad na presun zdravotníckeho materiálu, expresnú distribúciu drobných zásielok či dopravu kritických komponentov v priemyselných areáloch. Štúdie ukazujú, že drony môžu skrátiť čas doručenia v mestskom prostredí až o desiatky percent, najmä pri presune po zložitých dopravných trasách alebo v oblastiach s dopravnými zápchami.

Napriek tomu je transportná oblasť UAV stále v nedokonalom stave. Súčasné systémy trpia limitmi v oblasti energetickej efektivity, doletu, presnosti autonómnej navigácie, stability v náročných poveternostných podmienkach a koordinácie vo veľkom meradle. Kým koncepty UAM predpokladajú vytvorenie plnohodnotných „vzdušných koridorov“ nad mestami, súčasná technologická úroveň takúto prevádzku zatiaľ neumožňuje dlhodobo a bezpečne.

Štúdia zdôrazňuje, že úspech UAM závisí od prepojenia dronov s mestskými senzormi, jednotnej dátovej infraštruktúry, spoločných komunikačných štandardov, integrácie s dopravnými riadiacimi centrami, vytvorenia bezpečných a stabilných vzdušných trás. Taktiež poukazuje na dôležitosť integrovaných dátových platforiem, ktoré umožnia mestám monitorovať drony v reálnom čase, predchádzať kolíziám a efektívne riadiť ich pohyb v hustej zástavbe. Pre budúci rozvoj transportu je preto nevyhnutná úzka integrácia UAV so smart city infraštruktúrou, čo je jednou z vízií, ktorú projekt môže podporiť.




% týmto sa generuje zoznam literatúry z obsahu súboru literatura.bib podľa toho, na čo sa v článku odkazujete
\clearpage
\bibliographystyle{plain}
\bibliography{dopad}

\end{document}
