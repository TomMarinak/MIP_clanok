\documentclass[10pt,a4paper]{article}

\usepackage[slovak]{babel}
\usepackage[utf8]{inputenc}
\usepackage{graphicx}
\usepackage{amsmath}
\usepackage{hyperref}
\usepackage{cite}
\usepackage{acronym}
\usepackage[a4paper, top=2.5cm, bottom=2.5cm, left=2.5cm, right=2.5cm]{geometry}
\usepackage{array}  
\usepackage{booktabs}

\setlength{\parskip}{15pt}
\pagestyle{headings}

\begin{document}

\begin{center}
  {\Large\bfseries VV - F}\\[2pt]
  {\normalsize\bfseries VECNÝ ZÁMER PROJEKTU / MATERIAL INTENT OF THE PROJECT}
\end{center}
\vspace{8pt}
{\small \textit{VV 2024 – Záväzná osnova pre aplikovaný výskum a vývoj / Obligatory scheme for applied research and development. The obligatory scheme must not exceed 15 pages.}} 

\vspace{8pt}
\textbf{Project title / Názov projektu:} Masové používanie dronov na civilné účely

\vspace{6pt}
\textbf{Principal Investigator / Zodpovedný riešiteľ:} Vladimír Mlynarovič

\vspace{10pt}
\textbf{1. Excellence / Excelentnosť} \\[4pt]

\section{Excelentnosť projektu z vecného a odborného hľadiska}

\subsection{Projektový zámer}
Zámer projektu je jasne definovaný – vyvinúť decentralizovanú sieť dronov, ktoré dokážu samostatne spolupracovať a plniť úlohy ako doručovanie zásielok, sledovanie bezpečnostných incidentov či podpora humanitárnych operácií. Tento prístup predstavuje prelomové riešenie v oblasti mestskej vzdušnej mobility, keďže spája ekonomické, bezpečnostné a spoločenské aspekty do jedného integrovaného systému. Projekt reaguje na aktuálne globálne výzvy ako rastúca potreba automatizácie, boj proti kriminalite a rýchla reakcia v krízových situáciách.

\subsection{Metodika riešenia}
Metodika projektu je postavená na vytvorení decentralizovanej architektúry pozostávajúcej z mestských servisných staníc, autonómnych UAV jednotiek a riadiaceho systému využívajúceho umelú inteligenciu, senzorické moduly a moderné komunikačné siete. Tento prístup umožňuje dronom koordinovať lety bez centrálneho bodu riadenia, adaptívne reagovať na zmeny prostredia a efektívne distribuovať úlohy. Použitá metodika kombinuje inžiniersky návrh, simulačné modelovanie a testovanie algoritmov rozhodovania, čím zaručuje praktickú overiteľnosť a škálovateľnosť systému.

\subsection{Ciele projektu}
Hlavným cieľom projektu je navrhnúť a overiť funkčný koncept decentralizovanej siete dronov schopných samostatného rozhodovania a spolupráce. Medzi konkrétne ciele patria:
\begin{itemize}
    \item vývoj komunikačného a riadiaceho systému pre UAV flotilu,
    \item integrácia senzorických prvkov a autonómneho rozhodovania pomocou AI,
    \item simulácia a testovanie prevádzky v mestskom prostredí,
    \item návrh infraštruktúry servisných staníc pre údržbu a nabíjanie dronov,
    \item analýza možností nasadenia v civilnej logistike, bezpečnosti a humanitárnych misiách.
\end{itemize}
Tieto ciele sú konkrétne, merateľné a dosiahnuteľné v rámci akademického projektu, pričom poskytujú základ pre ďalší aplikovaný výskum a vývoj.

\subsection{Výstupy a závery}
Očakávaným výstupom projektu je návrh a simulačne overený model systému, ktorý dokáže efektívne koordinovať flotilu dronov pri vykonávaní viacerých typov úloh. Projekt prinesie technické riešenia, ktoré je možné priamo aplikovať v mestskom plánovaní, logistike, krízovom manažmente a civilnej ochrane. Závery projektu potvrdia uskutočniteľnosť decentralizovaného riadenia UAV s podporou umelej inteligencie a moderných sietí, čím sa otvorí priestor pre reálne pilotné implementácie.

\subsection{Kompetentnosť riešiteľov}
Projekt realizuje trojčlenný tím študentov Fakulty informatiky a informačných technológií STU v Bratislave. Každý člen tímu sa špecializuje na inú oblasť – umelú inteligenciu, sieťové technológie a softvérové inžinierstvo – čo zaručuje interdisciplinárny prístup a vysokú kvalitu spracovania. Projekt zároveň podporuje rozvoj mladých odborníkov v oblasti autonómnych systémov a inteligentnej dopravy, čím prispieva k budovaniu novej generácie inžinierov schopných riešiť komplexné technologické výzvy.

\subsection{Medzinárodný kontext}
Projekt MDCU má jasný medzinárodný presah. Decentralizovaný UAV systém je relevantný pre globálne iniciatívy v oblasti boja proti terorizmu, humanitárnej pomoci a riadenia katastrof v ťažko dostupných regiónoch. Navrhovaný model je kompatibilný s európskym rámcom U-space a reflektuje trendy vývoja inteligentnej vzdušnej mobility podľa odporúčaní EASA. Projekt tak spája lokálny výskum so svetovým kontextom a prispieva k medzinárodnej spolupráci v oblasti bezpečných a udržateľných UAV technológií.

\vspace{8pt}
\textbf{2. Impact / Dopad} \\

\section{DOPAD}

V následujúcej časti sa venujeme hodnoteniu dopadov masového nasadenia dronov a bezpilotných vzdušných prostriedkov (ďalej len UAV) na rôzne oblasti spoločenského, technologického a ekonomického života, so zameraním na ich prínosy pre infraštruktúru, logistiku a rozvoj moderných služieb, no taktiež na problemy s legislatívou a technickymi/fyzickými limitami dronov/UAV.

\subsection{Prínosy projektu na rozvoj poznania, aplikovaného výskumu a inovácií}
Projekt \textbf{MDCU} zameriava na detailné preskúmanie možností integrácie veľkého množstva dronov do mestského a prímestského prostredia, pričom projekt prináša nové postupy riadenia, navigácie a dohľadu nad autonómnymi letmi. Drony sa v súčasnosti používajú najmä individuálne, no masové využitie si vyžaduje nové vedecké prístupy, ktoré integrujú prvky umelej inteligencie, senzorických systémov a moderných komunikačných technológií.

Významným prínosom projektu je vývoj algoritmov pre koordináciu veľkého počtu dronov vo vzdušnom priestore. Ide najmä o plánovanie letových trás, predikciu kolíznych situácií, adaptívnu reakciu na prekážky a optimalizáciu energií. Projekt tak prispieva k vytváraniu tzv. U-space, čo je európsky koncept pre správu nízkoletového vzdušného priestoru. Zároveň rozvíja aj úplne nové prístupy v oblasti distribuovanej umelej inteligencie, kde jednotlivé drony spolupracujú ako inteligentný systém.

Inovačný význam je viditeľný aj v oblasti senzoriky. Projekt skúma nové druhy optických, termálnych a LIDAR senzorov optimalizovaných pre autonómny let, čo umožní bezpečnejšie operácie aj v komplikovaných podmienkach – nočné prostredie, mestské ulice, okolie infraštruktúry či miesta zníženej viditeľnosti.

Dôležitým výstupom je aj tvorba nových metodík pre implementáciu dronových služieb v každodenných civilných procesoch: doručovanie zásielok, monitoring kritickej infraštruktúry, automatizovaná dokumentácia, prvá pomoc, hodnotenie škôd po živelných udalostiach či zabezpečenie verejnej bezpečnosti. Tieto postupy môžu slúžiť ako štandardizované odporúčania pre firmy, mestá a štátne inštitúcie.

Projekt tak významne rozširuje súčasný stav poznania technologického aj spoločenského charakteru a tvorí pevný základ pre ďalší aplikovaný výskum v oblasti inteligentnej mobility.


\subsection{Spôsob a miera využiteľnosti výsledkov pre žiadateľa v SR a v zahraničí}
Výsledky projektu sú široko využiteľné pre organizácie pôsobiace v rôznych oblastiach. V Slovenskej republike je to najmä logistický sektor, poskytovatelia kuriérskych služieb, zdravotnícke zariadenia, záchranné jednotky a správcovia energetickej a dopravnej infraštruktúry. Pre tieto subjekty projekt predstavuje novú generáciu technológií, ktoré umožnia automatizovať procesy, znížiť náklady a zvýšiť bezpečnosť.

V oblasti samospráv môžu byť výsledky využité pri zavádzaní inteligentných mestských služieb, ktoré zahŕňajú napríklad pravidelný prehľad dopravy, monitorovanie životného prostredia, sledovanie výstavby alebo hodnotenie rizikových oblastí. Drony s pokročilými senzormi môžu zásadne zjednodušiť správu mestských území a priniesť efektívnejšie rozhodovanie úradov.

V zahraničí sú výsledky projektu kompatibilné s aktuálnymi štandardmi Európskej agentúry pre bezpečnosť letectva (EASA). To umožní plnú integráciu vyvinutých riešení do európskeho U-space a ich rýchle nasadenie v krajinách Európskej únie. Projekt preto podporuje export technológií a know-how, čím môže prispieť k posilneniu postavenia slovenských výskumných inštitúcií a firiem na medzinárodnej úrovni.

Potenciál využitia je však globálny – riešenia možno aplikovať v oblastiach s vysokou urbanizáciou, pri zásahoch v nedostupnom teréne či pri koordinácii humanitárnej pomoci. Projekt tak môže prispieť aj k zahraničným výskumným spoluprácam a môže sa stať základom ďalších medzinárodných iniciatív.


\subsection{Miera ekonomického prínosu}
Ekonomické prínosy projektu sú významné a rozmanité. V prvom rade ide o podporu technologického sektora v SR, ktorý môže priamo profitovať z vývoja nových zariadení, softvéru a analytických nástrojov. Firmy pôsobiace v oblasti hardvéru môžu využiť nové konštrukčné a senzorické riešenia, zatiaľ čo softvérové spoločnosti môžu stavať na algoritmoch a metódach riadenia dronov, ktoré projekt prináša.

Masové nasadenie dronov prináša výrazné ekonomické úspory. V logistike môžu drony skrátiť čas doručovania a znížiť potrebu časti pozemnej dopravy, čo vedie k zníženiu nákladov na palivo, servis a ľudské zdroje. V monitorovacích činnostiach nahrádzajú drony tradičné postupy založené na manuálnom zbieraní dát, vysielaní vozidiel či využití vrtuľníkov, ktoré sú finančne aj energeticky náročné.

Významný je aj prínos pre zamestnanosť. Aj keď drony automatizujú niektoré činnosti, zároveň vytvárajú nové profesie — operátor autonómnych systémov, analytik letových dát, dizajnér senzorických modulov, špecialista U-space infraštruktúry, servisný technik či odborník na kyberbezpečnosť vzdušných systémov. Projekt tak nepriamo podporuje vznik nových pracovných miest s vyššou pridanou hodnotou.

V neposlednom rade má projekt aj pozitívny environmentálny dopad. Vďaka elektrickému pohonu a vysokej efektivite dopravy prispievajú drony k znižovaniu emisií skleníkových plynov. Ekonomický prínos tu spočíva nielen v úsporách, ale aj v možnosti implementovať ekologickejšie procesy, ktoré sú v súlade s trendmi udržateľného rozvoja.


\subsection{Opatrenia na maximalizáciu výsledkov, dopadu a komunikáciu výstupov}
Aby boli výsledky projektu maximálne využité, je potrebné dôsledné plánovanie ich prenosu do praxe. Projekt počíta s intenzívnou komunikáciou smerom k výskumným inštitúciám, odborným partnerom aj verejnosti.

Výstupy projektu budú prezentované prostredníctvom odborných workshopov, konferencií a praktických demonštrácií, ktoré umožnia potenciálnym záujemcom priamo vidieť fungovanie vyvinutých riešení. Súčasťou komunikácie bude aj tvorba technickej dokumentácie, metodík pre mestá a poskytovateľov služieb, a rovnako aj tvorba otvorených datasetov, ktoré môžu ďalej využívať výskumníci.

Projekt počíta aj s vybudovaním partnerstiev so slovenskými aj zahraničnými organizáciami. Tieto partnerstvá umožnia rýchle zavedenie výsledkov do praxe a zabezpečia ich škálovanie. Súčasťou je aj koordinácia s regulačnými orgánmi, aby sa nové technológie mohli integrovať v súlade s legislatívou.

Dôležitou súčasťou maximalizácie dopadu je aj práca s verejnosťou. Keďže drony sú technológia s výrazným spoločenským dopadom, projekt bude aktívne vysvetľovať ich výhody, bezpečnostné mechanizmy a prínos pre spoločnosť. Transparentná komunikácia podporí dôveru verejnosti v nové služby, čo je kľúčové pre ich budúce nasadenie vo veľkom rozsahu.


\subsection{Transportné systémy dronov – súčasný stav a limity}
Transportná oblasť je jednou z najperspektívnejších aplikácií civilných dronov, avšak v praxi je stále v štádiu intenzívneho vývoja. Viaceré technologické prototypy už umožňujú prepravu menších zásielok na vzdialenosť niekoľkých kilometrov, no pre ich masové nasadenie je potrebné vyriešiť viacero technických aj prevádzkových problémov.

Najväčším obmedzením je kapacita batérií, ktorá limituje dolet aj hmotnosť prenášaného nákladu. Súčasné komerčné drony dokážu preniesť spravidla 2–5 kg, pričom ich letový čas sa pohybuje okolo 20–30 minút. Pre logistické spoločnosti je to nedostatočné na plnohodnotnú náhradu časti cestnej dopravy, najmä v rozľahlých alebo členitých oblastiach.

Ďalším problémom je spoľahlivá navigácia v komplexnom mestskom prostredí. Drony musia dynamicky reagovať na pohyb ľudí, automobilov či neplánované prekážky a zároveň sa riadiť presnými letovými koridormi. Systémy vizuálnej lokalizácie či LIDAR senzory dokážu tieto problémy čiastočne riešiť, no stále dochádza k nepresnostiam, najmä pri sťažených svetelných podmienkach alebo vo veľmi úzkych priestoroch medzi budovami.

Nezanedbateľným problémom je aj integrácia veľkého množstva transportných dronov do existujúcej vzdušnej dopravy. Na efektívnu prevádzku je potrebná spolahlivá koordinácia s U-space službami, predikcia kolíznych trajektórií, komunikácia medzi dronmi a infraštruktúrou a jednotné bezpečnostné protokoly. Tieto služby sú stále vo vývoji a zatiaľ nie sú plošne implementované.

Hoci sa transportné drony rýchlo približujú praktickému nasadeniu, ich používanie v rozsiahlej logistickej sieti je stále limitované technologickými rizikami, nevyspelosťou navigačných algoritmov a nedostatočnou robustnosťou v extrémnych podmienkach. Projekt preto prispieva k vývoju riešení, ktoré majú tieto problémy dlhodobo eliminovať.


\subsection{Legislatívne problémy spojené s prevádzkou dronov a UAV}
Masové používanie dronov v civilnom priestore naráža aj na významné legislatívne bariéry. Európska legislatíva je harmonizovaná prostredníctvom EASA, avšak mnoho pravidiel je stále relatívne nové a ich implementácia v členských štátoch nie je jednotná.

Jedným z hlavných problémov je kategorizácia rizikovosti letov. Pre viaceré plánované aplikácie, najmä pre autonómny transport zásielok nad obývanými oblasťami, je nutné spĺňať podmienky kategórie specific. To si vyžaduje vypracovanie detailného SORA(Specific Operations Risk Assessment) hodnotenia, ktoré je pre menšie organizácie finančne aj administratívne náročné. Mnohé startupy či municipalitné projekty tak nemajú kapacity vykonať všetky potrebné právne kroky.

Ďalšou výzvou je povinnosť udržiavať dron v priamom vizuálnom kontakte (VLOS) pri väčšine civilných operácií. Pre autonómne transportné operácie je však nevyhnutná prevádzka mimo priamej viditeľnosti (BVLOS). Hoci legislatíva BVLOS umožňuje, vyžaduje to špeciálne povolenia, redundantné systémy bezpečnosti (automatické detekčno-vyhýbacie systémy), a často aj dohľad operátora, čo výrazne komplikuje komerčné nasadenie.

Problémom je tiež ochrana súkromia a osobných údajov. Pri masovom nasadení dronov môže byť nevyhnutné monitorovať veľké územia pomocou kamier a senzorov, čo môže byť v rozpore s GDPR a národnými zákonmi o ochrane osobných údajov. Na zamedzenie neželaného zberu dát si legislatíva vyžaduje technické obmedzenia kamerových systémov, zónovanie letov či anonymizáciu záznamov, avšak tieto opatrenia môžu priamo ovplyvniť kvalitu navigácie a spoľahlivosť služieb.

V neposlednom rade legislatíva rieši aj hluk, rušenie rádiových frekvencií a zodpovednosť za škodu. Pri masových transportných operáciách je potrebné vytvoriť jednoznačné pravidlá pre prípady technických zlyhaní, straty spojenia, kolízie alebo pádu dronu. Súčasné predpisy však nie sú ešte úplne prispôsobené pre situácie, keď by v jednej oblasti mohli operovať desiatky či stovky autonómnych UAV súčasne.

Pre riešenie týchto problémov je potrebné úzke prepojenie technologického vývoja, štandardizácie a legislatívnych úprav. Projekt sa preto zameriava aj na návrh odporúčaní pre regulačné orgány, ktoré môžu v budúcnosti výrazne zjednodušiť proces licencovania a zabezpečiť bezpečné, efektívne a zákonne kompatibilné používanie dronov.

Rastúci trend urbanizácie a zvýšené nároky na logistiku vytvárajú tlak na zavádzanie nových foriem mobility. Moderné smart cities sa preto zameriavajú na koncept urban air mobility (UAM), ktorý využíva bezpilotné prostriedky na rýchly presun ľudí, tovarov a dát. Podľa vybraných medzinárodných štúdií, medzi nimi aj „Integrating Urban Air Mobility into Smart Cities"\cite{UAM-DiVito}, sa očakáva, že drony v najbližších desaťročiach prevezmú kľúčovú úlohu v mestskej logistike, monitorovaní a dynamickom riadení mestských procesov.

Transportné systémy dronov sa už začínajú objavovať v pilotných projektoch, kde slúžia napríklad na presun zdravotníckeho materiálu, expresnú distribúciu drobných zásielok či dopravu kritických komponentov v priemyselných areáloch. Štúdie ukazujú, že drony môžu skrátiť čas doručenia v mestskom prostredí až o desiatky percent, najmä pri presune po zložitých dopravných trasách alebo v oblastiach s dopravnými zápchami.

Napriek tomu je transportná oblasť UAV stále v nedokonalom stave. Súčasné systémy trpia limitmi v oblasti energetickej efektivity, doletu, presnosti autonómnej navigácie, stability v náročných poveternostných podmienkach a koordinácie vo veľkom meradle. Kým koncepty UAM predpokladajú vytvorenie plnohodnotných „vzdušných koridorov“ nad mestami, súčasná technologická úroveň takúto prevádzku zatiaľ neumožňuje dlhodobo a bezpečne.

Štúdia zdôrazňuje, že úspech UAM závisí od prepojenia dronov s mestskými senzormi, jednotnej dátovej infraštruktúry, spoločných komunikačných štandardov, integrácie s dopravnými riadiacimi centrami, vytvorenia bezpečných a stabilných vzdušných trás. Taktiež poukazuje na dôležitosť integrovaných dátových platforiem, ktoré umožnia mestám monitorovať drony v reálnom čase, predchádzať kolíziám a efektívne riadiť ich pohyb v hustej zástavbe. Pre budúci rozvoj transportu je preto nevyhnutná úzka integrácia UAV so smart city infraštruktúrou, čo je jednou z vízií, ktorú projekt môže podporiť.

\vspace{10pt}
\textbf{3. Implementation / Implementácia} \\

\section{IMPLEMENTÁCIA}

Implementácia projektu bude realizovaná v niekoľkých navzájom prepojených etapách, ktoré zabezpečia tvorbu funkčného demonštračného modelu decentralizovanej siete dronov (ďalej „systém“) určeného pre civilné účely — doručovanie, pomoc v núdzi, monitorovanie bezpečnosti. Cieľom je predviesť technickú aj organizačnú realizovateľnosť tejto koncepcie v mestskom prostredí.

\subsection*{Fáza 1: Analýza požiadaviek a špecifikácia systému}
V tejto úvodnej fáze bude vykonaná hĺbková analýza potrieb relevantných užívateľov a strán — logistických operátorov, záchranných služieb, bezpečnostných zložiek a mestských správcov. Zameriame sa na definovanie základných funkčných požiadaviek systému, ako sú: registrácia dronov a staníc (tzv. uzlov siete), prideľovanie úloh dronom, sledovanie polohy a stavu dronov v reálnom čase, výmena dát medzi uzlami a centrálnou kontrolou, adaptívne plánovanie letových trás a reakcia na dynamické udalosti (napr. núdzové zásahy). Pri tejto analýze budeme vychádzať aj z literatúry týkajúcej sa siete dronov a ich komunikačných a riadiacich výziev — napríklad survey článok „Unmanned Aerial Vehicles: A Survey on Civil Applications and Key Research Challenges“\cite{Shakhatreh:UAVSurvey} uvádza, že jedným z hlavných problémov nasadenia dronov v civilných aplikáciách je výdrž batérie, vyhýbanie sa kolíziám a komplexné sieťové a bezpečnostné požiadavky. 

\subsection*{Fáza 2: Návrh architektúry systému}
Na základe analytických výstupov bude navrhnutá architektúra systému, založená na decentralizovanom modeli riadenia siete dronov. Predpokladom je, že mesto je pokryté množstvom staníc (uzlov) pre drony — tieto uzly budú vybavené nabíjacími/preskladacími modulmi, komunikačnou infraštruktúrou a lokalizačnými senzormi. Drony budú spravované autonómne v rámci pridelených oblastí a budú schopné komunikovať medzi uzlami cez mesh-sieť alebo prostredníctvom mobilnej siete (napr. 5G) ako záložnej infraštruktúry — čo korešponduje s prácou „A Survey on Cellular-connected UAVs: Design Challenges, Enabling 5G/B5G Innovations, and Experimental Advancements“\cite{Mishra:CellularUAV}. 
Komunikačný model bude definovaný tak, že uzly môžu prijímať a odosielať informácie o pozícii, stave, úlohách a prioritách dronov v reálnom čase. Taktiež bude navrhnuté rozloženie letových zón, priemerné vzdialenosti dronov od staníc, režim nabíjania a výmenu dronov na staniciach.  
Na modelovanie architektúry systému budú použité nástroje ako UML (use-case diagramy, komponentné diagramy) alebo nástroje vizualizácie (napr. Draw.io) a simulácie návrhov v prostredí Python, pričom budú využité knižnice pre simuláciu udalostí (napr. SimPy) a vizualizáciu trás a prevádzky. Tým sa overí funkčnosť návrhu pred implementáciou prototypu.

\subsection*{Fáza 3: Vývoj prototypu a simulácia prevádzky}
V tejto fáze bude vyvinutý demonštračný prototyp softvérovej časti systému. Prototyp bude pozostávať z komponentov: satelitný model alebo simulátor dronov, riadiaceho uzla (server) a jednoduchého používateľského rozhrania (web- alebo mobilná aplikácia) pre zadávanie úloh a vizualizáciu prevádzky. Simulácia bude demonštrovať spracovanie viacerých dronov súčasne v mestskej oblasti, prideľovanie úloh, nabíjanie a výmenu dronov na staniciach, reakciu na núdzové situácie a preladenie prevádzky podľa časových a priestorových parametrov. Výstup simulácie bude vizualizácia trás, využitia zdrojov (staníc, dronov), a analýza metrík ako priemerný čas doručenia/zásahu, percento využitia staníc, počet úloh za jednotku času. Tento prístup vychádza aj z literatúry „Networked Unmanned Aerial Vehicles for Surveillance and Monitoring: A Survey“\cite{Holland:NetworkedUAV}, ktorá skúma komunikačné modely a požiadavky na siete dronov pre civilné monitorovanie.

Simulácia tiež umožní testovanie scenárov: masové doručenia zásielok počas špičky, paralelné zásahy pri rôznych udalostiach, a robustnosť systému pri výpadku jednej alebo viacerých staníc. Na základe výsledkov simulácie budú navrhnuté odporúčania pre veľkosť siete staníc, počet dronov, režim výmeny nabíjania a optimálne rozloženie uzlov.

\subsection*{Fáza 4: Pilotné nasadenie a testovanie v reálnom prostredí}
Po úspešnej simulácii bude navrhnutý plán pilotného nasadenia v obmedzenom mestskom území – napríklad v konkrétnej mestskej časti. Vybraný segment bude slúžiť na testovanie systému v reálnych podmienkach – doručenie malých zásielok, zásah dronu v simulovanej núdzovej situácii (napr. doručenie výbavy prvej pomoci) a monitorovanie vybranej oblasti pomocou siete dronov. Pilotné nasadenie umožňuje reálne overenie komunikácie, reakčného času, bezpečnostných aspektov a koordinácie medzi dronmi a uzlami. Bude definované monitorovanie výkonnosti systému – napríklad spoľahlivosť komunikácie, úspešnosť vykonania úloh, využitie batérií, výpadky dronov či staníc, latencia riadenia. V tejto fáze budú vyriešené aj otázky integrácie s existujúcou infraštruktúrou (napr. logistické centrá, zásahové zložky) a budú definované prevádzkové postupy a bezpečnostné protokoly.
Zohľadnené budú aj legislatívne aspekty nasadenia dronovej siete v meste – koordinácia s letovými službami, pravidlá nasadzovania dronov nad urbanizovaným územím, ochrana osobných údajov a bezpečnostné opatrenia.

\subsection*{Fáza 5: Vyhodnotenie, optimalizácia a dokumentácia}
Na záver bude vykonané vyhodnotenie pilotného nasadenia – zber dát o prevádzke systému, analýza výkonnosti a identifikácia úzkych miest (napr. stanice s nábehom na preťaženie, oblasť s nízkym pokrytím, nadmerné čakanie dronov na výmenu batérie). Na základe týchto dát bude realizovaná optimalizácia — úprava počtu dronov a staníc, prepracovanie atribučného modelu prideľovania úloh, vylepšenie komunikačných protokolov a plánovanie letových trás. Dokumentácia bude zahŕňať technickú správu, používateľskú príručku pre systém, návrh správy prevádzky siete a odporúčania pre škálovanie systému v širšom mestskom prostredí.

\section{Rozpočet}

\begin{tabular}{|>{\raggedright}p{5cm}|>{\raggedright}p{7cm}|r|}
\hline
\textbf{Položka} & \textbf{Množstvo / popis} & \textbf{Odhad nákladu (EUR)} \\
\hline
Drony (5 ks) & civilné viacúčelové drony, vrátane senzora a komunikácie & 12000 \\
\hline
Nabíjacie/prenášacie stanice (3 ks) & modulárne uzly so stanicou na výmenu batérií + sieťové pripojenie & 6000 \\
\hline
Komunikačná infraštruktúra & router/mesh-uzly, SIM/konektivita 5G pre testovacie nasadenie & 3000 \\
\hline
Vývoj softvéru a simulácie & licencie, vývojári, testovanie (2 mesiace práce) & 8000 \\
\hline
Legislatíva/súlad s normami & konzultácie, povolenia, bezpečnostné protokoly & 1500 \\
\hline
Pilotné nasadenie & prenájom priestoru, testovacie scenáre & 2500 \\
\hline
Rezerva (cca 10\%) & nepredvídané výdavky & 3000 \\
\hline
\textbf{Celkom} & & \textbf{36000} \\
\hline
\end{tabular}

\noindent\textit{Poznámka: Rozpočet je orientačný a bude doladený v rámci ďalšej fázy projektu.}



% týmto sa generuje zoznam literatúry z obsahu súboru literatura.bib podľa toho, na čo sa v článku odkazujete
\clearpage
\bibliographystyle{plain}
\bibliography{vvf-zamer}

\end{document}
