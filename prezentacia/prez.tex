% Valentino Vranic
% Metody inzinierskej prace 2012/13

\documentclass{beamer}

%\usetheme{Warsaw}
\usetheme{Antibes}
%\usetheme{JuanLesPins}
%\usetheme{Goettingen}

%\usecolortheme{seahorse}
%\usecolortheme{dolphin}
\usecolortheme{rose}
% https://deic.uab.cat/~iblanes/beamer_gallery/
%\usecolortheme{beaver}

%\useoutertheme[]{sidebar}

\setbeamercovered{transparent}

\usepackage[slovak]{babel}
\usepackage[T1]{fontenc}
\usepackage[utf8]{inputenc}
\usepackage{url}
\usepackage{graphicx}

\usepackage{listings}

\lstset{language=C++,basicstyle=\fontsize{8}{9.6}\selectfont,showstringspaces=false,columns=fullflexible,identifierstyle=\ttfamily,keywordstyle=\bfseries,showstringspaces=false,columns=fullflexible}
%\lstset{language=C,basicstyle=\fontsize{10.5}{12.6}\selectfont,identifierstyle=\ttfamily,keywordstyle=\bfseries,showstringspaces=false,columns=fixed}

\def\BibTeX{\textsc{Bib}\kern-.08em\TeX} 

\newcommand{\footcite}[1]{\footnote{\tiny #1}}
\newcommand{\umlet}{.5}
\newcommand{\emp}[1]{\textit{\alert{#1}}}
\newcommand{\kw}[1]{\mbox{\textbf{#1}}}
\newcommand{\id}[1]{\texttt{#1}}
\newcommand{\stl}{\guillemotleft}
\newcommand{\str}{\guillemotright}

\newcommand{\lsti}{\lstinline[basicstyle=\fontsize{10.5}{12.1}\selectfont]}

\newcommand{\ssection}[1]{
	\section{#1}
	\begin{frame}[fragile=singleslide]\frametitle{}
	\Huge #1
	\end{frame}
}

\newcommand{\ssectionn}[1]{
	\section*{#1}
	\begin{frame}[fragile=singleslide]\frametitle{}
	\Huge #1
	\end{frame}
}

\newenvironment{program}{\begin{beamercolorbox}[rounded=true,shadow=true]{block body}\vspace{-4mm}}{\vspace{-2mm}\end{beamercolorbox}}

\setbeamercolor{fvystup}{fg=white,bg=black}
\newenvironment{vystup}{\begin{beamercolorbox}[rounded=true,shadow=true]{fvystup}}{\end{beamercolorbox}}

\newenvironment{poznamka}{\begin{beamercolorbox}[rounded=true,shadow=false]{block body}}{\end{beamercolorbox}}

\setbeamertemplate{footline}[page number]
{
%\insertpagenumber
%\begin{beamercolorbox}{section in head/foot}
%\vskip2pt\insertnavigation{\paperwidth}\vskip2pt
%\end{beamercolorbox}%
}



\author{Tomáš Marinňák}
%\url{www.fiit.stuba.sk/~vranic}, \url{vranic@fiit.stuba.sk}}
%{\tiny \url{www.fiit.stuba.sk/~vranic}, \url{vranic@fiit.stuba.sk}}
\institute{
	Fakulta informatiky a informačných technológií\\
	Slovenská technická univerzita v Bratislave}

\subtitle{\vspace{3mm} Metódy inžinierskej práce 2019/2020}

\title{Príklad prezentácie
}

\date{\footnotesize 5. november 2020}




\begin{document}

\begin{frame}[fragile=singleslide]
\titlepage
\end{frame}


\begin{frame}[fragile=singleslide]\frametitle{O čom to je}
Ešte pre prehľadom prezentácie sa zvyčajne uvádza motivácia.

Bežne sa aj tu používajú odrážky, ale tento text je naschvál uvedený bez odrážok. Niekedy môže byť potrebné uviesť aj citáty\ldots{}

Toto je len príklad slajdov. Ako urobiť dobrú prezentáciu bolo vysvetlené na prednáške.
\end{frame}


\begin{frame}[fragile=singleslide]\frametitle{Prehľad}
\tableofcontents
\end{frame}


\section{Časť 1}
% príkaz \ssection by vytvoril zvláštný slajd s názvom časti - v krátkych prezentáciách to prekáža, lebo oberá o čas

\begin{frame}[fragile=singleslide]\frametitle{Nejaký slajd}
\begin{itemize}
\item Odrážky na úrovni 1
\item 2. odražka na urovni 1
	\begin{itemize}
	\item ukažka odražky na urovni 2
	\item Ďalšia  na urovni 2
	\end{itemize}
\item zasa uroveň 1
\end{itemize}
\end{frame}



\section{Časť 2}

\begin{frame}[fragile=singleslide]\frametitle{Ďalší slajd}
\begin{itemize}
\item Text
\item Ďalší text -- \emph{zvýraznený text}
\item \emp{Key note} % príkaz definovaný v preambule

% odrážka s odkazom na zdroj:
\item Bol použitý balík beamer\footcite{\url{https://fiit.stuba.sk}}
\end{itemize}
\end{frame}


\section{Časť 3}

\begin{frame}[fragile=singleslide]\frametitle{Rámiky}
\begin{poznamka}
Text možno uviesť v rámiku
\end{poznamka}
\end{frame}

\section{Časť 4}
\begin{frame}[fragile=singleslide]\frametitle{Ďalší slajd}

\end{frame}

\section*{Zhodnotenie a ďalšia práca}
% hviezdička zabezpečí, aby sa táto časť neocitla v prehľade prezentácie - každá prezentácia má zhodnotenie a prehľad by sa tým zbytočne zahlcoval

\begin{frame}[fragile=singleslide]\frametitle{Zhodnotenie a ďalšia práca}
\begin{itemize}
\item Každá prezentácia musí byť nejako uzavretá
\item Ale vždy je čo robiť ďalej\ldots{}
\end{itemize}
\end{frame}

\end{document}




Text \end{document} za príkazom \end{document} LaTeX ignoruje, takže tu môžete odkladať veci (aj celé slajdy), ktoré nechcete vymazať, lebo ich ešte možno budete potrebovať, avšak ich v danom momente nechcete mať v slajdoch.
