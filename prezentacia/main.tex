\documentclass[aspectratio=169]{beamer}

\usepackage[T2A]{fontenc}
\usepackage[utf8]{inputenc}
\usepackage[slovak,english]{babel}

\usetheme{Madrid}
\setbeamertemplate{navigation symbols}{}
\setbeamertemplate{footline}[frame number] 


\title{Masové používanie dronov na civilné účely}
\author{Dmytro Seredenko}
\date{November 2025}

\begin{document}

\begin{frame}
    \centering
    {\small SEREDENKO \quad MARINAK \quad LUKSAJ}
    
    \vspace{0.8cm}
    {\LARGE MASOVÉ POUŽÍVANIE DRONOV}\\[0.2cm]
    {\LARGE NA CIVILNÉ ÚČELY}
    \vfill
    {\small Presentation}
    \vspace{0.3cm}
    {\small 18/11/2025}
\end{frame}

\begin{frame}{Introduction}
    \vspace{0.5cm}
    \begin{columns}[T]
        \column{0.48\textwidth}
        \begin{itemize}
            \item Rýchly rast UAV technológií
            \item Výzvy legislatívy, infraštruktúry a spoločenskej akceptácie
        \end{itemize}
        \column{0.48\textwidth}
        \begin{itemize}
            \item Potreba efektívnej a bezpečnej koordinácie flotíl
            \item Cieľ projektu: návrh, analýza a overenie systému pre masové nasadenie UAV
        \end{itemize}
    \end{columns}
\end{frame}

\begin{frame}{Ciele projektu}
\begin{columns}[T]
    \column{0.33\textwidth}
    \textbf{Zjednodušenie logistiky}
    \vspace{0.2cm}
    \begin{itemize}
        \item Rýchlejšie doručovanie zásielok
        \item Flexibilnejšie logistické riešenia
        \item Efektívne doručovanie v mestách
    \end{itemize}
    \column{0.33\textwidth}
    \textbf{Zjednodušenie zabezpečovania bezpečnosti}
    \vspace{0.2cm}
    \begin{itemize}
        \item Neustále monitorovanie území
        \item Rýchlejšia reakcia bezpečnostných zložiek
        \item Zvýšená efektivita zásahov
    \end{itemize}
    \column{0.33\textwidth}
    \textbf{Bezpečnejšie pracovné podmienky}
    \vspace{0.2cm}
    \begin{itemize}
        \item Znižovanie rizika pre pracovníkov
        \item Kontrola konštrukcií a nebezpečných zón
        \item Nasadenie v rizikových povolaniach
    \end{itemize}

\end{columns}

\end{frame}

\begin{frame}{Rast trhu s dronmi}
\begin{figure}
    \centering
    \includegraphics[width=0.7\linewidth]{image.png}
    \vspace{0.4cm}
    \small
    
    Rast globálneho trhu s dronmi (2018–2030) podľa regiónov.
\end{figure}
\end{frame}

\begin{frame}{Oblasti využitia}
\begin{figure}
    \centering
    \includegraphics[width=0.6\linewidth]{image2.png}
    \vspace{0.2cm}
    \small
    
    Najčastejšie oblasti využitia a ciele využitia
\end{figure}
\end{frame}

\begin{frame}{Ekonomický prínos}
\begin{columns}[T]
    \column{0.33\textwidth}
    \textbf{Kommerčný prínos pre súkromný sektor}
    \vspace{0.2cm}
    \begin{itemize}
        \item Rýchla a efektívna preprava malých zásielok
        \item Zníženie nákladov pre firmy
        \item Podpora e-commerce a expresných služieb
    \end{itemize}
    \column{0.33\textwidth}
    \textbf{Kommerčný prínos pre štát}
    \vspace{0.2cm}
    \begin{itemize}
        \item Lacnejšia prevádzka oproti personálu v teréne
        \item Minimalizácia rizík v nebezpečných situáciách
        \item Dlhodobé zníženie štátnych výdavkov
    \end{itemize}
    \column{0.33\textwidth}
    \textbf{Zjednodušenie kontroly nad mestským prostredím}
    \vspace{0.2cm}
    \begin{itemize}
        \item Zlepšený dohľad nad infraštruktúrou
        \item Rýchle identifikovanie hrozieb a porúch
        \item Posilnenie bezpečnosti a stability mesta
    \end{itemize}
\end{columns}
\end{frame}

\begin{frame}{Štruktúrny koncept systému}

	\centering
    \includegraphics[width=1\linewidth]{deployment_diagram.png}
    \vspace{0.4cm}
    \small

\end{frame}

\begin{frame}{Perspektívy realizácie}

    \textbf{Technologické obmedzenia}
    \begin{itemize}
	\item { Súčasné UAV systémy sú dosť technicky limitované }
	\item { }
    \end{itemize}

    \vspace{0.8cm}

    \textbf{Možné riešenia a perspektíva technologického vývoja}

\end{frame}

\begin{frame}{Etické otázky v oblasti masového nasadenia dronov}
\begin{columns}[T]
    \column{0.48\textwidth}
    \textbf{Ochrana súkromia a právo na anonymitu}
    
    \vspace{0.2cm}
    Masové nasadenie UAV prináša riziko neúmyselného alebo nadmerného zbierania obrazových
    a lokalizačných údajov o obyvateľoch.
    \column{0.48\textwidth}
    \textbf{Zodpovednosť a bezpečnosť autonómnych rozhodnutí}
    
    \vspace{0.2cm}
    Autonómne UAV systémy musia v reálnom čase robiť rozhodnutia,
    ktoré môžu ovplyvniť bezpečnosť ľudí, majetku či dopravy.
    Otázkou zostáva, kto nesie zodpovednosť v prípade zlyhania — výrobca,
    operátor alebo samotný systém.
\end{columns}
\end{frame}

\begin{frame}{Príklady a úspešné prípady využitia dronov}
\textbf{Amazon Prime Air}

\vspace{0.3cm}
Amazon používa drony na doručovanie malých balíkov zákazníkom do 30 minút.
Tento systém umožňuje rýchlejšie doručovanie, znižuje dopravnú záťaž
a poskytuje zákazníkom pohodlnú službu. Drony sú vybavené senzormi
a GPS, aby bezpečne doručili balík priamo k dverám zákazníka.
\end{frame}

\begin{frame}{Budúcnosť dronov s umelou inteligenciou}
\begin{columns}[T]
    \column{0.55\textwidth}
    \textbf{UI a drony}
    
    \vspace{0.2cm}
    Umelá inteligencia umožní dronom lietať autonómne, analyzovať dáta
    v reálnom čase a koordinovať sa vo formácii „roju“. Drony dokážu samy
    plánovať optimálne trasy doručania, predchádzať nebezpečenstvám
    a monitorovať poľnohospodárske plochy či infraštruktúru efektívnejšie
    než človek. Už dnes sa využíva pri doručovaní balíkov, sledovaní polí
    a pri bezpečnostných operáciách.
    \column{0.45\textwidth}
\end{columns}

\end{frame}



\end{document}
